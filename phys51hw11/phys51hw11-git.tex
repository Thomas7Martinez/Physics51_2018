

\documentclass[11pt]{article}
\include{header} % Download this file from physics.hmc.edu/motion/ page
\usepackage{tikz}

\usetikzlibrary{arrows,calc}
                	\tikzset{%
                         every picture/.style={>=stealth'},
                                        vel/.style={->,line width=2pt,color=DarkBlue},
                  force/.style={line width=1.5pt,color=blue,->},
                  coord/.style={color=green!40!black,|->},
                  accel/.style={->,line width=3pt,color=gray},
                  photon/.style={line width=1.5pt,color=DarkRed,decorate,decoration={snake,post length=0.1in}},
                  spring/.style={decorate,decoration={coil,aspect=0.3,segment length=2mm,amplitude=2mm}},
                  traj/.style={dashed, color=gray, line width=1pt}
                }
                
\usepackage{fullpage}
\setlength{\parskip}{6pt}
\setlength{\parindent}{0pt}
\usepackage[margin=1in]{geometry}
\usepackage{graphicx}
\usepackage{enumerate}
\usepackage{marvosym}
\usepackage{amssymb}
\usepackage{wasysym}
\usepackage{gensymb}
\usepackage{mathrsfs}
\usepackage{scrextend}
\usepackage{mathtools}
\usepackage{pgfplots}
\usepackage{xspace}
\usepackage[colorlinks]{hyperref}

\pgfplotsset{compat=1.15}
% --- style --- %
\renewcommand{\labelenumi}{{ (\alph{enumi})}}
\newcommand{\sand}{\quad \mbox{ and } \quad}
\newcommand{\be}{\begin{enumerate}[a) ]}
\newcommand{\ee}{\end{enumerate}}
\def\bal#1\eal{\begin{align*}#1\end{align*}}
\allowdisplaybreaks

% --- making \xi look less awful --- %
\DeclareSymbolFont{CMletters}{OML}{cmm}{m}{it}
\DeclareMathSymbol{\xi}{\mathord}{CMletters}{"18}

% --- math --- %
\newcommand{\Z}{\mathbb{Z}}
\newcommand{\R}{\mathbb{R}}
\newcommand{\C}{\mathbb{C}}
\newcommand{\Q}{\mathbb{Q}}


\newcommand{\Lt}[1]{\mathcal{L}\crb{#1}}
\newcommand{\ilt}[1]{\mathcal{L}^{-1}\crb{#1}}

\newcommand{\pn}[1]{\left( #1 \right)}
\newcommand{\sqb}[1]{\left[ #1 \right]}
\newcommand{\crb}[1]{\left\{ #1 \right\}}
\newcommand{\lra}[1]{\left\langle #1 \right\rangle}
\newcommand{\magn}[1]{\left\lVert #1 \right\rVert}

\def\multiset#1#2{\ensuremath{\left(\kern-.3em\left(\genfrac{}{}{0pt}{}{#1}{#2}\right)\kern-.3em\right)}}

\newcommand{\pdr}[2]{\frac{\partial #1}{\partial #2}}
\newcommand{\pdrr}[2]{\frac{\partial^2 #1}{\partial #2^2}}
\newcommand{\im}[1]{\text{im}\pn{#1}}
\newcommand{\m}[1]{\Z/#1\Z}


\DeclareMathOperator{\proj}{proj}
\newcommand{\vectorproj}[2][]{\proj_{\VEC{#1}}\VEC{#2}}

\newenvironment{amatrix}[1]{%
  \left(\begin{array}{@{}*{#1}{c}|c@{}}
}{%
  \end{array}\right)
}

\makeatletter
\renewcommand*\env@matrix[1][*\c@MaxMatrixCols c]{%
  \hskip -\arraycolsep
  \let\@ifnextchar\new@ifnextchar
  \array{#1}}
\makeatother

\newcommand{\spn}[1]{\text{span}\pn{#1}}

\newcommand*\Heq{\ensuremath{\overset{\kern2pt H}{=}}}

\newcommand{\distil}{\sqrt{1-v^2/c^2}}
\newcommand{\distilf}[1]{\sqrt{1-(#1)^2}}
\newcommand{\lorentz}{\frac{1}{\distil}}
\newcommand{\lorentzf}[1]{\frac{1}{\sqrt{1-(#1)^2}}}


\begin{document}

\noindent{\large Problem Set 11, 3 December 2018\hfill Name: \underline{\hspace{3cm}} ,  Section: \underline{\hspace{5mm}} }
\vspace*{0.25in}


\begin{problem}[(P38.14)*]

The figure below shows a parallel-plate capacitor being charged.
\be
\item Show that the Poynting vector $\VEC{S}$ points everywhere radially into the cylindrical volume.
\item Show that the rate at which energy flows into this volume, calculated by integrating the
Poynting vector over the cylindrical boundary of this volume, is equal to the rate at which
the stored electrostatic energy increases; that is,
\[
	\int \mathbf { S } \cdot d \mathbf { A } = A d \frac { d } { d t } \left( \frac { 1 } { 2 } \epsilon _ { 0 } E ^ { 2 } \right)
\]
where $Ad$ is the volume of the capacitor and $\frac{1}{2}\epsilon_0E^2$
is the energy density for all points within
that volume.
\ee
This analysis shows that, according to the Poynting vector point of view, the energy stored in
a capacitor does not enter it through the wires but through the space around the wires and the
plates. (Hint: To find $\VEC{S}$ we must first find $\VEC{B}$, which is the magnetic field set up by the displacement
current during the charging process; see the figure below. Ignore fringing of the lines of $\VEC{E}$.)
\begin{center}
\includegraphics[scale=0.4]{prob1.png}
\end{center}

\end{problem}


% Your solution starts here %%%%%%%%%%%%%%%%%%%%%%%%%%%%%%%%%%%%%%%%%%%%%%%%%%
\textbf{Solution:}\\

\clearpage
% Your solution ends here %%%%%%%%%%%%%%%%%%%%%%%%%%%%%%%%%%%%%%%%%%%%%%%%%%

\begin{problem}[(E38.25) (\textit{3 points})]
The intensity of direct solar radiation not absorbed by the atmosphere on a particular summer day is $\val{130}{W/m}^2$. 
How close would you have to stand to a 1.0 kW electric heater to feel the same
intensity? Assume that the heater radiates uniformly in all directions.
\end{problem}


% Your solution starts here %%%%%%%%%%%%%%%%%%%%%%%%%%%%%%%%%%%%%%%%%%%%%%%%%%
\textbf{Solution:}\\

\clearpage
% Your solution ends here %%%%%%%%%%%%%%%%%%%%%%%%%%%%%%%%%%%%%%%%%%%%%%%%%%

\begin{problem}[(E38.28) (\textit{2 points})]
Sunlight strikes the Earth, just outside its atmosphere, with an intensity of 1.38 kW/$\text{m}^2$. Calculate
\be
\item $E_m$ and
\item $B_m$ for sunlight, assuming it to be a plane wave.
\ee
\end{problem}


% Your solution starts here %%%%%%%%%%%%%%%%%%%%%%%%%%%%%%%%%%%%%%%%%%%%%%%%%%
\textbf{Solution:}\\

\clearpage
% Your solution ends here %%%%%%%%%%%%%%%%%%%%%%%%%%%%%%%%%%%%%%%%%%%%%%%%%%

\begin{problem}[Supplementary Problem 6]
Light traveling in air ($n_1 = 1$) enters the smooth, flat surface of a pond ($n_2 = 1.33$) at normal
incidence.
\be
\item What fraction of the light is reflected and what fraction is transmitted?
\item  If the maximum amplitude of the electric field in the incident light is $E_0$, what is the maximum
amplitude of the electric field in the reflected light?
\item When you see scenery reflected in a still pond or lake, how is that situation different from
the one you have just calculated?
\ee
\end{problem}


% Your solution starts here %%%%%%%%%%%%%%%%%%%%%%%%%%%%%%%%%%%%%%%%%%%%%%%%%%
\textbf{Solution:}\\

\clearpage
% Your solution ends here %%%%%%%%%%%%%%%%%%%%%%%%%%%%%%%%%%%%%%%%%%%%%%%%%%

\begin{problem}[(P44.3)]
A stack of polarizing sheets is arranged so that the angle between any two adjacent sheets is $\alpha$.
The sheets are arranged so that $N$ sheets rotate the plane of polarization by $\theta$, where $\theta = N \alpha$.
Calculate the fraction of light that will pass through the stack in the limit as $N \to \infty$. Assume
that $\theta$ is fixed, so $\alpha \to 0$.
\end{problem}


% Your solution starts here %%%%%%%%%%%%%%%%%%%%%%%%%%%%%%%%%%%%%%%%%%%%%%%%%%
\textbf{Solution:}\\

\clearpage
% Your solution ends here %%%%%%%%%%%%%%%%%%%%%%%%%%%%%%%%%%%%%%%%%%%%%%%%%%

\begin{problem}[(P44.4)]
It is desired to rotate the plane of vibration of a beam of polarized light by $90^{\circ}$.
\be
\item How might this be done using only polarizing sheets?
\item How many sheets are required for the total intensity loss to be less than 5.0\%?
\ee
\end{problem}


% Your solution starts here %%%%%%%%%%%%%%%%%%%%%%%%%%%%%%%%%%%%%%%%%%%%%%%%%%
\textbf{Solution:}\\

\clearpage
% Your solution ends here %%%%%%%%%%%%%%%%%%%%%%%%%%%%%%%%%%%%%%%%%%%%%%%%%%

\begin{problem}[(E44.3)]
A beam of unpolarized light of intensity 12.2 mW/$\text{m}^2$
falls at normal incidence on a polarizing
sheet.
\be
\item Find the maximum value of the electric field of the transmitted beam.
\item Calculate the radiation pressure exerted on the polarizing sheet.
\ee
\end{problem}


% Your solution starts here %%%%%%%%%%%%%%%%%%%%%%%%%%%%%%%%%%%%%%%%%%%%%%%%%%
\textbf{Solution:}\\

\clearpage
% Your solution ends here %%%%%%%%%%%%%%%%%%%%%%%%%%%%%%%%%%%%%%%%%%%%%%%%%%

\begin{problem}[Townsend (P1.4) \P]
A radio station broadcasts at a frequency $\nu = \val{91.5}{MHz}$ with a total radiated power of $P = \val{20}{kW}$.
\be
\item What is the wavelength $\lambda$ of this radiation?
\item What is the energy of each photon in eV? How many photons are emitted each second? How
many photons are emitted in each cycle?
\item A particular radio receiver requires 2.0 microwatts of radiation to provide intelligible reception.
How many 91.5 MHz photons does this require per second? per cycle?
\item  Do the answers to (b) and (c) indicate that the granularity of the electromagnetic radiation
can be neglected in these circumstances?
\ee
\end{problem}


% Your solution starts here %%%%%%%%%%%%%%%%%%%%%%%%%%%%%%%%%%%%%%%%%%%%%%%%%%
\textbf{Solution:}\\

\clearpage
% Your solution ends here %%%%%%%%%%%%%%%%%%%%%%%%%%%%%%%%%%%%%%%%%%%%%%%%%%

\begin{problem}[Townsend (P1.9) \P \textit{(3 points)}]
A beam of UV light of wavelength $\lambda = \val{197.0}{nm}$ falls onto a metal cathode. The stopping potential
needed to keep any electrons from reaching the anode is 2.08 V.
\be
\item What is the work function $W$ of the cathode surface, in eV?
\item What is the velocity $v$ of the fastest electrons emitted from the cathode? \textit{Note}: Since
$K_{\text{max}}/mc^2 \ll 1$, the nonrelativistic expression for the kinetic energy can be utilized here.
\item If Avogadro's number of photons strikes each square meter of the surface in one hour, what
is the average intensity $I$ of the beam, in units of W/$\text{m}^2$?
\ee
\end{problem}


% Your solution starts here %%%%%%%%%%%%%%%%%%%%%%%%%%%%%%%%%%%%%%%%%%%%%%%%%%
\textbf{Solution:}\\

\clearpage
% Your solution ends here %%%%%%%%%%%%%%%%%%%%%%%%%%%%%%%%%%%%%%%%%%%%%%%%%%

\begin{problem}[Townsend (P1.13) \P \textit{(2 points)}]
The maximum kinetic energy of electrons ejected from sodium is 1.85 eV for radiation of 300 nm
and 0.82 eV for radiation of 400 nm. Use this data to determine Planck's constant and the work
function of sodium.
\end{problem}


% Your solution starts here %%%%%%%%%%%%%%%%%%%%%%%%%%%%%%%%%%%%%%%%%%%%%%%%%%
\textbf{Solution:}\\

\clearpage
% Your solution ends here %%%%%%%%%%%%%%%%%%%%%%%%%%%%%%%%%%%%%%%%%%%%%%%%%%

\end{document}