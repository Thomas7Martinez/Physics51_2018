

\documentclass[11pt]{article}
\include{header} % Download this file from physics.hmc.edu/motion/ page
\usepackage{tikz}

\usetikzlibrary{arrows,calc}
                	\tikzset{%
                         every picture/.style={>=stealth'},
                                        vel/.style={->,line width=2pt,color=DarkBlue},
                  force/.style={line width=1.5pt,color=blue,->},
                  coord/.style={color=green!40!black,|->},
                  accel/.style={->,line width=3pt,color=gray},
                  photon/.style={line width=1.5pt,color=DarkRed,decorate,decoration={snake,post length=0.1in}},
                  spring/.style={decorate,decoration={coil,aspect=0.3,segment length=2mm,amplitude=2mm}},
                  traj/.style={dashed, color=gray, line width=1pt}
                }
                
\usepackage{fullpage}
\setlength{\parskip}{6pt}
\setlength{\parindent}{0pt}
\usepackage[margin=1in]{geometry}
\usepackage{graphicx}
\usepackage{enumerate}
\usepackage{marvosym}
\usepackage{amssymb}
\usepackage{wasysym}
\usepackage{gensymb}
\usepackage{mathrsfs}
\usepackage{scrextend}
\usepackage{mathtools}
\usepackage{pgfplots}
\usepackage{xspace}
\usepackage[colorlinks]{hyperref}

\pgfplotsset{compat=1.15}
% --- style --- %
\renewcommand{\labelenumi}{{ (\alph{enumi})}}
\newcommand{\sand}{\quad \mbox{ and } \quad}
\newcommand{\be}{\begin{enumerate}[a) ]}
\newcommand{\ee}{\end{enumerate}}
\def\bal#1\eal{\begin{align*}#1\end{align*}}
\allowdisplaybreaks

% --- making \xi look less awful --- %
\DeclareSymbolFont{CMletters}{OML}{cmm}{m}{it}
\DeclareMathSymbol{\xi}{\mathord}{CMletters}{"18}

% --- math --- %
\newcommand{\Z}{\mathbb{Z}}
\newcommand{\R}{\mathbb{R}}
\newcommand{\C}{\mathbb{C}}
\newcommand{\Q}{\mathbb{Q}}


\newcommand{\Lt}[1]{\mathcal{L}\crb{#1}}
\newcommand{\ilt}[1]{\mathcal{L}^{-1}\crb{#1}}

\newcommand{\pn}[1]{\left( #1 \right)}
\newcommand{\sqb}[1]{\left[ #1 \right]}
\newcommand{\crb}[1]{\left\{ #1 \right\}}
\newcommand{\lra}[1]{\left\langle #1 \right\rangle}
\newcommand{\magn}[1]{\left\lVert #1 \right\rVert}

\def\multiset#1#2{\ensuremath{\left(\kern-.3em\left(\genfrac{}{}{0pt}{}{#1}{#2}\right)\kern-.3em\right)}}

\newcommand{\pdr}[2]{\frac{\partial #1}{\partial #2}}
\newcommand{\pdrr}[2]{\frac{\partial^2 #1}{\partial #2^2}}
\newcommand{\im}[1]{\text{im}\pn{#1}}
\newcommand{\m}[1]{\Z/#1\Z}


\DeclareMathOperator{\proj}{proj}
\newcommand{\vectorproj}[2][]{\proj_{\VEC{#1}}\VEC{#2}}

\newenvironment{amatrix}[1]{%
  \left(\begin{array}{@{}*{#1}{c}|c@{}}
}{%
  \end{array}\right)
}

\makeatletter
\renewcommand*\env@matrix[1][*\c@MaxMatrixCols c]{%
  \hskip -\arraycolsep
  \let\@ifnextchar\new@ifnextchar
  \array{#1}}
\makeatother

\newcommand{\spn}[1]{\text{span}\pn{#1}}

\newcommand*\Heq{\ensuremath{\overset{\kern2pt H}{=}}}

\newcommand{\distil}{\sqrt{1-v^2/c^2}}
\newcommand{\distilf}[1]{\sqrt{1-(#1)^2}}
\newcommand{\lorentz}{\frac{1}{\distil}}
\newcommand{\lorentzf}[1]{\frac{1}{\sqrt{1-(#1)^2}}}


\begin{document}

\noindent{\large Problem Set 13, 12 December 2018\hfill Name: \underline{\hspace{3cm}} ,  Section: \underline{\hspace{5mm}} }
\vspace*{0.25in}



\begin{problem}[Townsend (P1.37)*]
Determine the probability that a photon is detected at the first minimum of a six-slit grating if the
bottom two slits are closed. Assume the magnitude of the probability amplitude due to each slit is
$r$. \textit{Suggestion}: Start by showing how the complex probability amplitudes from each slit add up to
zero at the first minimum.
\end{problem}


% Your solution starts here %%%%%%%%%%%%%%%%%%%%%%%%%%%%%%%%%%%%%%%%%%%%%%%%%%
\textbf{Solution:}\\

\clearpage
% Your solution ends here %%%%%%%%%%%%%%%%%%%%%%%%%%%%%%%%%%%%%%%%%%%%%%%%%%

\begin{problem}[Townsend (P1.43)]
Use the principle of least time to derive Snell's law, namely, $n_1 \sin \theta_1 = n_2 \sin \theta_2$ for light being
refracted as it travels from a medium with index of refraction $n_1$ into a medium with index of
refraction $n_2$. \textit{Suggestion}: Follow a procedure similar to the one given in Example 1.11. Locate
the source S in medium 1 and the point P in medium 2.
\end{problem}


% Your solution starts here %%%%%%%%%%%%%%%%%%%%%%%%%%%%%%%%%%%%%%%%%%%%%%%%%%
\textbf{Solution:}\\

\clearpage
% Your solution ends here %%%%%%%%%%%%%%%%%%%%%%%%%%%%%%%%%%%%%%%%%%%%%%%%%%


\end{document}