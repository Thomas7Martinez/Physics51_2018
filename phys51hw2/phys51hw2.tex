\documentclass[11pt]{article}
\include{header} % Download this file from physics.hmc.edu/motion/ page
\usepackage{tikz}

\usetikzlibrary{arrows,calc}
                	\tikzset{%
                         every picture/.style={>=stealth'},
                                        vel/.style={->,line width=2pt,color=DarkBlue},
                  force/.style={line width=1.5pt,color=blue,->},
                  coord/.style={color=green!40!black,|->},
                  accel/.style={->,line width=3pt,color=gray},
                  photon/.style={line width=1.5pt,color=DarkRed,decorate,decoration={snake,post length=0.1in}},
                  spring/.style={decorate,decoration={coil,aspect=0.3,segment length=2mm,amplitude=2mm}},
                  traj/.style={dashed, color=gray, line width=1pt}
                }
                
\usepackage{fullpage}
\setlength{\parskip}{6pt}
\setlength{\parindent}{0pt}
\usepackage[margin=1in]{geometry}
\usepackage{graphicx}
\usepackage{enumerate}
\usepackage{marvosym}
\usepackage{amssymb}
\usepackage{wasysym}
\usepackage{gensymb}
\usepackage{mathrsfs}
\usepackage{scrextend}
\usepackage{mathtools}
\usepackage{pgfplots}
\usepackage{xspace}
\usepackage[colorlinks]{hyperref}

% --- style --- %
\renewcommand{\labelenumi}{{ (\alph{enumi})}}
\newcommand{\sand}{\quad \mbox{ and } \quad}
%\newcommand{\ds}{\displaystyle}
\allowdisplaybreaks

% --- making \xi look less awful --- %
\DeclareSymbolFont{CMletters}{OML}{cmm}{m}{it}
\DeclareMathSymbol{\xi}{\mathord}{CMletters}{"18}

% --- math --- %
\newcommand{\Z}{\mathbb{Z}}
\newcommand{\R}{\mathbb{R}}
\newcommand{\C}{\mathbb{C}}
\newcommand{\Q}{\mathbb{Q}}


\newcommand{\Lt}[1]{\mathcal{L}\crb{#1}}
\newcommand{\ilt}[1]{\mathcal{L}^{-1}\crb{#1}}

\newcommand{\pn}[1]{\left( #1 \right)}
\newcommand{\sqb}[1]{\left[ #1 \right]}
\newcommand{\crb}[1]{\left\{ #1 \right\}}
\newcommand{\lra}[1]{\left\langle #1 \right\rangle}
\newcommand{\magn}[1]{\left\lVert #1 \right\rVert}

\newcommand{\pdr}[2]{\frac{\partial #1}{\partial #2}}
\newcommand{\im}[1]{\text{im}\pn{#1}}
\newcommand{\m}[1]{\Z/#1\Z}

\DeclareMathOperator{\proj}{proj}
\newcommand{\vectorproj}[2][]{\proj_{\VEC{#1}}\VEC{#2}}

\newenvironment{amatrix}[1]{%
  \left(\begin{array}{@{}*{#1}{c}|c@{}}
}{%
  \end{array}\right)
}

\newcommand{\spn}[1]{\text{span}\pn{#1}}

\newcommand*\Heq{\ensuremath{\overset{\kern2pt H}{=}}}

\newcommand{\distil}{\sqrt{1-v^2/c^2}}
\newcommand{\distilf}[1]{\sqrt{1-(#1)^2}}
\newcommand{\lorentz}{\frac{1}{\distil}}
\newcommand{\lorentzf}[1]{\frac{1}{\sqrt{1-(#1)^2}}}

\begin{document}

\noindent{\large Problem Set 2, 17 September 2018\hfill Name: \underline{\hspace{3cm}} ,  Section: \underline{\hspace{5mm}} }
\vspace*{0.25in}


\begin{problem}[(P27.3)*]
A small sphere whose mass $m$ is \val{1.12}{mg} carries a charge $q=\val{19.7}{nC}$. It hangs in the Earth's gravitational field from a silk thread that makes an
angle $\theta = 27.4\degree$ with a large, uniformly, charged, nonconducting sheet as seen in the figure below. Calculate the uniform charge density $\sigma$ for the sheet.
\begin{center}
\includegraphics[scale=0.5]{prob1.png}
\end{center}
\end{problem}


% Your solution starts here %%%%%%%%%%%%%%%%%%%%%%%%%%%%%%%%%%%%%%%%%%%%%%%%%%
\textbf{Solution:}

% Your solution ends here %%%%%%%%%%%%%%%%%%%%%%%%%%%%%%%%%%%%%%%%%%%%%%%%%%

\clearpage
\begin{problem}[(E27.7)]
A point charge $+q$ is a distance $d/2$ from a square surface of side $d$ and is directly above the center of a square, as shown in the figure below. Find the electric flux through
the square. (Hint: Think of the square as one face of a cube with edge $d$.)
\begin{center}
\includegraphics[scale=0.75]{prob2.png}
\end{center}
\end{problem}


% Your solution starts here %%%%%%%%%%%%%%%%%%%%%%%%%%%%%%%%%%%%%%%%%%%%%%%%%%
\textbf{Solution:}

% Your solution ends here %%%%%%%%%%%%%%%%%%%%%%%%%%%%%%%%%%%%%%%%%%%%%%%%%%

\clearpage

\begin{problem}[(P27.16)]
A plane slab of thickness $d$ has a uniform volume charge density $\rho$. Find the magnitude of the electric field at all points in space both (a) inside and (b) outside
the slab, in terms of $x$, the distance measured from the median plane of the slab.
\end{problem}


% Your solution starts here %%%%%%%%%%%%%%%%%%%%%%%%%%%%%%%%%%%%%%%%%%%%%%%%%%
\textbf{Solution:}

% Your solution ends here %%%%%%%%%%%%%%%%%%%%%%%%%%%%%%%%%%%%%%%%%%%%%%%%%%

\clearpage

\begin{problem}[(P27.17)]
A solid nonconducting sphere of radius $R$ carries a nonuniform charge distribution, with charge density $\rho = \rho_S r/R$, where $\rho_S$ is a constant
and $r$ is the distance from the center of the sphere. Show that (a) the total charge on the sphere is $Q = \pi\rho_S R^3$ and (b) the electric field inside the
sphere is given by
$$
	E = \frac{1}{4\pi\epsilon_0} \frac{Q}{R^4} r^2.
$$
\end{problem}


% Your solution starts here %%%%%%%%%%%%%%%%%%%%%%%%%%%%%%%%%%%%%%%%%%%%%%%%%%
\textbf{Solution:}

% Your solution ends here %%%%%%%%%%%%%%%%%%%%%%%%%%%%%%%%%%%%%%%%%%%%%%%%%%

\clearpage

\begin{problem}[(P27.4)]
The figure below shows a charge $+q$ arranged as a uniform conducting sphere of radius $a$ and placed at the center of a spherical conducting shell of inner radius $b$ and outer
radius $c$. The outer shell carries a charge of $-q$. Find $E(r)$ at locations (a) within the sphere $(r<a)$, (b) between the sphere and the shell $(a<r<b)$, (c) inside the shell $(b<r<c)$, and (d) outside the shell $(r>c)$. (e) What charges appear on the inner and outer surfaces of the shell?
\begin{center}
\includegraphics[scale=0.6]{prob5.png}
\end{center}
\end{problem}


% Your solution starts here %%%%%%%%%%%%%%%%%%%%%%%%%%%%%%%%%%%%%%%%%%%%%%%%%%
\textbf{Solution:}

% Your solution ends here %%%%%%%%%%%%%%%%%%%%%%%%%%%%%%%%%%%%%%%%%%%%%%%%%%

\clearpage

\begin{problem}[\P (E29.29)]
A 1-$\mu C$ point charge is embedded in the center of a solid Pyrex sphere of radius $R=\val{10}{cm}$. (a) Calculate the electric field strength $E$ just beneath the surface of the sphere. (b) Assuming that there are no other \textit{free} charges, calculate the strength of the electric field just outside the surface on the sphere. (c) What is the induced surface
charge density $\sigma_{\text{ind}}$ on the surface of the Pyrex sphere?
\end{problem}


% Your solution starts here %%%%%%%%%%%%%%%%%%%%%%%%%%%%%%%%%%%%%%%%%%%%%%%%%%
\textbf{Solution:}

% Your solution ends here %%%%%%%%%%%%%%%%%%%%%%%%%%%%%%%%%%%%%%%%%%%%%%%%%%

\clearpage

\end{document}