\documentclass[11pt]{article}
\include{header} % Download this file from physics.hmc.edu/motion/ page
\usepackage{tikz}

\usetikzlibrary{arrows,calc}
                	\tikzset{%
                         every picture/.style={>=stealth'},
                                        vel/.style={->,line width=2pt,color=DarkBlue},
                  force/.style={line width=1.5pt,color=blue,->},
                  coord/.style={color=green!40!black,|->},
                  accel/.style={->,line width=3pt,color=gray},
                  photon/.style={line width=1.5pt,color=DarkRed,decorate,decoration={snake,post length=0.1in}},
                  spring/.style={decorate,decoration={coil,aspect=0.3,segment length=2mm,amplitude=2mm}},
                  traj/.style={dashed, color=gray, line width=1pt}
                }
                
\usepackage{fullpage}
\setlength{\parskip}{6pt}
\setlength{\parindent}{0pt}
\usepackage[margin=1in]{geometry}
\usepackage{graphicx}
\usepackage{enumerate}
\usepackage{marvosym}
\usepackage{amssymb}
\usepackage{wasysym}
\usepackage{gensymb}
\usepackage{mathrsfs}
\usepackage{scrextend}
\usepackage{mathtools}
\usepackage{pgfplots}
\usepackage{xspace}
\usepackage[colorlinks]{hyperref}

% --- style --- %
\renewcommand{\labelenumi}{{ (\alph{enumi})}}
\newcommand{\sand}{\quad \mbox{ and } \quad}
%\newcommand{\ds}{\displaystyle}
\allowdisplaybreaks

% --- making \xi look less awful --- %
\DeclareSymbolFont{CMletters}{OML}{cmm}{m}{it}
\DeclareMathSymbol{\xi}{\mathord}{CMletters}{"18}

% --- math --- %
\newcommand{\Z}{\mathbb{Z}}
\newcommand{\R}{\mathbb{R}}
\newcommand{\C}{\mathbb{C}}
\newcommand{\Q}{\mathbb{Q}}


\newcommand{\Lt}[1]{\mathcal{L}\crb{#1}}
\newcommand{\ilt}[1]{\mathcal{L}^{-1}\crb{#1}}

\newcommand{\pn}[1]{\left( #1 \right)}
\newcommand{\sqb}[1]{\left[ #1 \right]}
\newcommand{\crb}[1]{\left\{ #1 \right\}}
\newcommand{\lra}[1]{\left\langle #1 \right\rangle}
\newcommand{\magn}[1]{\left\lVert #1 \right\rVert}

\newcommand{\pdr}[2]{\frac{\partial #1}{\partial #2}}
\newcommand{\im}[1]{\text{im}\pn{#1}}
\newcommand{\m}[1]{\Z/#1\Z}

\DeclareMathOperator{\proj}{proj}
\newcommand{\vectorproj}[2][]{\proj_{\VEC{#1}}\VEC{#2}}

\newenvironment{amatrix}[1]{%
  \left(\begin{array}{@{}*{#1}{c}|c@{}}
}{%
  \end{array}\right)
}

\newcommand{\spn}[1]{\text{span}\pn{#1}}

\newcommand*\Heq{\ensuremath{\overset{\kern2pt H}{=}}}

\newcommand{\distil}{\sqrt{1-v^2/c^2}}
\newcommand{\distilf}[1]{\sqrt{1-(#1)^2}}
\newcommand{\lorentz}{\frac{1}{\distil}}
\newcommand{\lorentzf}[1]{\frac{1}{\sqrt{1-(#1)^2}}}

\begin{document}

\noindent{\large Problem Set 3, 24 September 2018\hfill Name: \underline{\hspace{3cm}} ,  Section: \underline{\hspace{5mm}} }
\vspace*{0.25in}


\begin{problem}[(P28.6)*]
A particle of mass $m$, charge $q>0$, and initial kinetic energy $K$ is projected (from an infinite separation) toward a heavy nucleus of charge $Q$, 
assumed to have a fixed position in our reference frame.
\begin{enumerate}
\item[(a) ]  If the aim is ``perfect," how close to the center of the nucleus is the particle when it comes instantaneously to rest?
\item[(b) ]  With a particular imperfect aim, the particle's closest approach to the nucleus is twice the distance determined in part (a). Determine the speed of the particle at this closest distance of approach. Assume that the particle does not reach the surface of the nucleus.
\end{enumerate}
\end{problem}


% Your solution starts here %%%%%%%%%%%%%%%%%%%%%%%%%%%%%%%%%%%%%%%%%%%%%%%%%%
\textbf{Solution:}

% Your solution ends here %%%%%%%%%%%%%%%%%%%%%%%%%%%%%%%%%%%%%%%%%%%%%%%%%%

\clearpage
\begin{problem}[2.]
Consider an infinitely long cylindrical rod of radius $R$ with a volume charge density $\rho = a\,r,$ for $r\leq R$.
\begin{enumerate}
\item[(a) ]  Find $\VEC{E}$ in all regions of space, inside and outside the cylinder.
\item[(b) ]  Show that your results are consistent with the differential form of Gauss's Law.
\end{enumerate}
\end{problem}


% Your solution starts here %%%%%%%%%%%%%%%%%%%%%%%%%%%%%%%%%%%%%%%%%%%%%%%%%%
\textbf{Solution:}

% Your solution ends here %%%%%%%%%%%%%%%%%%%%%%%%%%%%%%%%%%%%%%%%%%%%%%%%%%

\clearpage

\begin{problem}[(E28.27)]
Two charges $q = +\val{2.13\,\mu}{C}$ are fixed in space a distance $d = \val{1.96}{cm}$ apart, as shown in the figure below.
\begin{enumerate}
\item[(a) ] What is the electric potential at point C? Take $V=0$ at infinity.
\item[(b) ] You bring a third charge $Q = +\val{1.91\,\mu}{C}$ slowly from infinity to C. How much work do you do?
\item[(c) ] What is the potential energy $U$ of the configuration when the third charge is in place?
\end{enumerate}
\begin{center}
\includegraphics[scale=0.5]{prob3.png}
\end{center}
\end{problem}


% Your solution starts here %%%%%%%%%%%%%%%%%%%%%%%%%%%%%%%%%%%%%%%%%%%%%%%%%%
\textbf{Solution:}

% Your solution ends here %%%%%%%%%%%%%%%%%%%%%%%%%%%%%%%%%%%%%%%%%%%%%%%%%%

\clearpage

\begin{problem}[(E28.42)]
Consider two widely separated conducting spheres, 1 and 2, the second having twice the diameter of the first. The smaller sphere initially
has a positive charge $q$ and the larger one is initially uncharged. You now connect the spheres with a very long thin wire.
\begin{enumerate}
\item[(a) ] How are the final potentials $V_1$ and $V_2$ of the spheres related?
\item[(b) ] Find the final charges $q_1$ and $q_2$ on the spheres in terms of $q$.
\end{enumerate}
\end{problem}


% Your solution starts here %%%%%%%%%%%%%%%%%%%%%%%%%%%%%%%%%%%%%%%%%%%%%%%%%%
\textbf{Solution:}

% Your solution ends here %%%%%%%%%%%%%%%%%%%%%%%%%%%%%%%%%%%%%%%%%%%%%%%%%%

\clearpage

\begin{problem}[(\P 5.)]
Consider a sphere of radius $R$ with a constant volume charge density $\rho$. Derive expressions for the electric field $\VEC{E}(r)$ and the electric potential
$V(r)$ for all regions of space, $r\leq R$ and $r\geq R$. Then sketch graphs of these functions aligned on the same $r$ scale. Let $V(\infty)=0$.
\end{problem}


% Your solution starts here %%%%%%%%%%%%%%%%%%%%%%%%%%%%%%%%%%%%%%%%%%%%%%%%%%
\textbf{Solution:}

% Your solution ends here %%%%%%%%%%%%%%%%%%%%%%%%%%%%%%%%%%%%%%%%%%%%%%%%%%

\clearpage

\begin{problem}[6.]
Before Rutherford's discovery of the atomic nucleus in
1911, a prevalent model of the atom was J. J. Thomson's
``plum pudding" model, consisting of negative electrons
distributed within a diffuse uniform cloud of positive
charge. In this archaic model, a helium atom might look
like the illustration below, with a spherical cloud of radius $R$
and charge $+2Q$ enclosing two electrons of charge $-Q$
each. Using your results from Problem 5, find the net
potential energy of the electrons and then determine the
equilibrium spacing $x$ that minimizes the potential
energy. (\textit{Note:} An equivalent alternative to using $V(r)$
from Problem 5 is to find the value of $x$ for which the net
force on the electrons is zero.)
\begin{center}
\includegraphics[scale=0.5]{prob6.png}
\end{center}
\end{problem}


% Your solution starts here %%%%%%%%%%%%%%%%%%%%%%%%%%%%%%%%%%%%%%%%%%%%%%%%%%
\textbf{Solution:}

% Your solution ends here %%%%%%%%%%%%%%%%%%%%%%%%%%%%%%%%%%%%%%%%%%%%%%%%%%

\clearpage

\end{document}