\documentclass[11pt]{article}
\include{header} % Download this file from physics.hmc.edu/motion/ page
\usepackage{tikz}

\usetikzlibrary{arrows,calc}
                	\tikzset{%
                         every picture/.style={>=stealth'},
                                        vel/.style={->,line width=2pt,color=DarkBlue},
                  force/.style={line width=1.5pt,color=blue,->},
                  coord/.style={color=green!40!black,|->},
                  accel/.style={->,line width=3pt,color=gray},
                  photon/.style={line width=1.5pt,color=DarkRed,decorate,decoration={snake,post length=0.1in}},
                  spring/.style={decorate,decoration={coil,aspect=0.3,segment length=2mm,amplitude=2mm}},
                  traj/.style={dashed, color=gray, line width=1pt}
                }
                
\usepackage{fullpage}
\setlength{\parskip}{6pt}
\setlength{\parindent}{0pt}
\usepackage[margin=1in]{geometry}
\usepackage{graphicx}
\usepackage{enumerate}
\usepackage{marvosym}
\usepackage{amssymb}
\usepackage{wasysym}
\usepackage{gensymb}
\usepackage{mathrsfs}
\usepackage{scrextend}
\usepackage{mathtools}
\usepackage{pgfplots}
\usepackage{xspace}
\usepackage[colorlinks]{hyperref}

% --- style --- %
\renewcommand{\labelenumi}{{ (\alph{enumi})}}
\newcommand{\sand}{\quad \mbox{ and } \quad}
\newcommand{\be}{\begin{enumerate}[a) ]}
\newcommand{\ee}{\end{enumerate}}
\def\bal#1\eal{\begin{align*}#1\end{align*}}
\allowdisplaybreaks

% --- making \xi look less awful --- %
\DeclareSymbolFont{CMletters}{OML}{cmm}{m}{it}
\DeclareMathSymbol{\xi}{\mathord}{CMletters}{"18}

% --- math --- %
\newcommand{\Z}{\mathbb{Z}}
\newcommand{\R}{\mathbb{R}}
\newcommand{\C}{\mathbb{C}}
\newcommand{\Q}{\mathbb{Q}}


\newcommand{\Lt}[1]{\mathcal{L}\crb{#1}}
\newcommand{\ilt}[1]{\mathcal{L}^{-1}\crb{#1}}

\newcommand{\pn}[1]{\left( #1 \right)}
\newcommand{\sqb}[1]{\left[ #1 \right]}
\newcommand{\crb}[1]{\left\{ #1 \right\}}
\newcommand{\lra}[1]{\left\langle #1 \right\rangle}
\newcommand{\magn}[1]{\left\lVert #1 \right\rVert}

\def\multiset#1#2{\ensuremath{\left(\kern-.3em\left(\genfrac{}{}{0pt}{}{#1}{#2}\right)\kern-.3em\right)}}

\newcommand{\pdr}[2]{\frac{\partial #1}{\partial #2}}
\newcommand{\pdrr}[2]{\frac{\partial^2 #1}{\partial #2^2}}
\newcommand{\im}[1]{\text{im}\pn{#1}}
\newcommand{\m}[1]{\Z/#1\Z}


\DeclareMathOperator{\proj}{proj}
\newcommand{\vectorproj}[2][]{\proj_{\VEC{#1}}\VEC{#2}}

\newenvironment{amatrix}[1]{%
  \left(\begin{array}{@{}*{#1}{c}|c@{}}
}{%
  \end{array}\right)
}

\makeatletter
\renewcommand*\env@matrix[1][*\c@MaxMatrixCols c]{%
  \hskip -\arraycolsep
  \let\@ifnextchar\new@ifnextchar
  \array{#1}}
\makeatother

\newcommand{\spn}[1]{\text{span}\pn{#1}}

\newcommand*\Heq{\ensuremath{\overset{\kern2pt H}{=}}}

\newcommand{\distil}{\sqrt{1-v^2/c^2}}
\newcommand{\distilf}[1]{\sqrt{1-(#1)^2}}
\newcommand{\lorentz}{\frac{1}{\distil}}
\newcommand{\lorentzf}[1]{\frac{1}{\sqrt{1-(#1)^2}}}


\begin{document}

\noindent{\large Problem Set 10, 19 November 2018\hfill Name: \underline{\hspace{3cm}} ,  Section: \underline{\hspace{5mm}} }
\vspace*{0.25in}


\begin{problem}[Supplementary Problem 4]
In a material of non-zero electrical resisitivity $\rho$, the relationship between electric field and current
density is $\VEC{E} = \rho\VEC{j}$. For copper, $\rho = 2 \times 10^{-8} \Omega m$. A copper wire with a circular cross-sectional area
of $4 \text{ mm}^2$ carries a current of 40 A.
\be
\item What is the longitudinal electric field (field along the length of the wire) in the copper?
\item If the current is changing at a rate of 5000 A/s, at what rate is $\VEC{E}$ changing, and what is the
resulting displacement current?
\item Does the displacement current contribute significantly to the magnetic field outside the wire?
Explain your answer.
\ee
\end{problem}


% Your solution starts here %%%%%%%%%%%%%%%%%%%%%%%%%%%%%%%%%%%%%%%%%%%%%%%%%%
\textbf{Solution:}\\

\clearpage
% Your solution ends here %%%%%%%%%%%%%%%%%%%%%%%%%%%%%%%%%%%%%%%%%%%%%%%%%%

\begin{problem}[(E38.16)*]
The electric field associated with a plane electromagnetic wave is given by $E_x = 0, E_y = 0,
E_z = E_0 \sin k(x-ct)$, where $E_0 = 2.34 \times 10^{-4}$ V/m and $k = 9.72 \times 10^6$/m. The wave is propagating
in the $+x$ direction.
\be
\item Write expressions for the components of the magnetic field of the wave.
\item Find the wavelength of the wave.
\ee
\end{problem}


% Your solution starts here %%%%%%%%%%%%%%%%%%%%%%%%%%%%%%%%%%%%%%%%%%%%%%%%%%
\textbf{Solution:}\\

\clearpage
% Your solution ends here %%%%%%%%%%%%%%%%%%%%%%%%%%%%%%%%%%%%%%%%%%%%%%%%%%

\begin{problem}[Supplementary Problem 5]
\be
\item Consider an electromagnetic wave in a vacuum with electric field $\VEC{E} = E_0\VEC{\hat{y}} \sin(kx-\omega t)$. What
is the propagation direction of this electromagnetic wave?
\item Consider an electromagnetic wave with electric field $\VEC{E} = E_0\VEC{\hat{y}} \sin(kx + \omega t)$. What is the
propagation direction of this electromagnetic wave?
\item Consider the electric field $\VEC{E} = E_0\VEC{\hat{y}} [\sin(kx - \omega t) + \sin(kx + \omega t)]$. Show that this electric field
satisfies the wave equation
\[
	\pdrr{\VEC{E}}{x} + \pdrr{\VEC{E}}{y} + \pdrr{\VEC{E}}{z} = \frac{1}{V^2}\pdrr{\VEC{E}}{t},
\]
provided the constants $k$ and $\omega$ are related as in part (a).
\ee
\end{problem}


% Your solution starts here %%%%%%%%%%%%%%%%%%%%%%%%%%%%%%%%%%%%%%%%%%%%%%%%%%
\textbf{Solution:}\\

\clearpage
% Your solution ends here %%%%%%%%%%%%%%%%%%%%%%%%%%%%%%%%%%%%%%%%%%%%%%%%%%

\begin{problem}[\P(P38.5) \textit{(3 points)}]
A cube of edge $a$ has its edges parallel to the $x$, $y$, and $z$ axes of a rectangular coordinate system.
A uniform electric field $\VEC{E}$ is parallel to the $y$ axis and a uniform magnetic field $\VEC{B}$ is parallel to the
$x$ axis. Calculate
\be
\item the rate at which, according to the Poynting vector point of view, energy may be said to pass
through each face of the cube and
\item the net rate at which the energy stored in the cube may be said to change.
\ee
\end{problem}


% Your solution starts here %%%%%%%%%%%%%%%%%%%%%%%%%%%%%%%%%%%%%%%%%%%%%%%%%%
\textbf{Solution:}\\

\clearpage
% Your solution ends here %%%%%%%%%%%%%%%%%%%%%%%%%%%%%%%%%%%%%%%%%%%%%%%%%%

\begin{problem}[\P(E38.22) \textit{(2 points)}]
A plane electromagnetic wave is traveling in the negative $y$ direction. At a particular position and
time, the magnetic field is along the positive $z$ axis and has a magnitude of 28 nT. What are the
direction and magnitude of the electric field at that position and at that time?
\end{problem}


% Your solution starts here %%%%%%%%%%%%%%%%%%%%%%%%%%%%%%%%%%%%%%%%%%%%%%%%%%
\textbf{Solution:}\\

\clearpage
% Your solution ends here %%%%%%%%%%%%%%%%%%%%%%%%%%%%%%%%%%%%%%%%%%%%%%%%%%

\begin{problem}[\P (P38.13)]
A plane electromagnetic wave, with wavelength 3.18 m, travels in free space in the $+x$ direction
with its electric vector $\VEC{E}$, of amplitude 288 V/m, directed along the $y$ axis.
\be
\item What is the frequency of the wave?
\item What is the direction and amplitude of the magnetic field associated with the wave?
\item If $\VEC{E} = E_m \sin(kx - \omega t)$, what are the values of $k$ and $\omega$?
\item Find the intensity of the wave.
\item If the wave falls on a perfectly absorbing sheet of area $1.85 \text{ m}^2$, at what rate would momentum
be delivered to the sheet and what is the radiation pressure exerted on the sheet?
\ee
\end{problem}


% Your solution starts here %%%%%%%%%%%%%%%%%%%%%%%%%%%%%%%%%%%%%%%%%%%%%%%%%%
\textbf{Solution:}\\

\clearpage
% Your solution ends here %%%%%%%%%%%%%%%%%%%%%%%%%%%%%%%%%%%%%%%%%%%%%%%%%%


\end{document}


